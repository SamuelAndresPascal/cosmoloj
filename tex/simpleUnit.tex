\documentclass[a4paper,draft,twoside,10pt]{article}

\usepackage[francais]{babel}
\usepackage{fontspec}
\usepackage{comment}
\usepackage{tikz-uml}
\usepackage{algorithm}
\usepackage{algpseudocode}

%\usepackage{amsmath}

% Compiler avec:
% xelatex simpleUnit

\title{Simple Unit\\0.1-r0}
\author{Samuel Andrés}

\begin{document}

\maketitle

\section{Versionnement et recommandations}

Le numéro de version de se document est composé d'un numéro de version majeure, d'un numéro de version mineure et d'un
numéro de révision:

<version majeure>.<version mineure>-r<révision>

Un changement de numéro de version majeure intervient lors d'une modification du contrat cassant la rétrocompatibilité
avec les versions précédentes.

Un changement de numéro de version mineure intervient lors d'une modification du contrat qui ne casse pas la
rétrocompatibilité.

Un changement de numéro de version de révison intervient à tout autre modification de ce document sans effet sur le
contrat \emph{ou} relevant de corrections d'erreurs flagrantes.

Les changements de version majeure, mineure et de révision interviennent chacune par incrémentation d'une valeur
entière. Le numéro de révision est réinitialisé à 0 lors d'un changement de version mineure ou majeure. Le numéro de
version mineure est réinitialisé à 0 lors d'un changement de version majeure.

On \emph{recommande} que les implémentations se référant à cette spécification suivent elles-mêmes un versionnement à
trois valeurs (majeure, mineure et révision) les versions majeure et mineure se référant respectivement aux versions
majeure et mineure de la présente spécification et la version de révision étant réservée aux correctifs internes de
l'implémentation ou aux modifications qui ne relèvent pas de critères impératifs de la présente spécification (par
exemple, si une implémentation simplement \emph{suggérée} par la présente spécification vient à disparaître d'une
implémentation, quoique cela crée ainsi une rupture de rétrocompatibilité). On \emph{recommande} aussi dans ce cas que
l'implémentation indique de manière explicite que son numéro de version se réfère au numéro de version de la présente
spécification. Dans le cas contraire, toute implémentation se référant à la présente spécification sans s'aligner sur
ses numéros de version majeure et de version mineure \emph{doit} indiquer explicitement le numéro de version complet de
la spécification.





\section{Spécification abstraite}

\subsection{Périmètre}

La présente spécification abstraite s'inspire de la JSR 363 de la communauté Java mais en restreint le périmètre.

Elle est destinée à guider l'implémentation de bibliothèques de construction et de conversion d'unités et non des
collections d'unités courante ou exaustive. Il ne s'agit pas non plus d'une bibliothèque d'interprétation de chaines de
caractères comme unités ni d'écriture d'unités sous forme de chaînes de caractères.

En définissant les unes par rapport aux autres les unités qui lui sont nécessaires, la présente spécification vise à
permettre à l'utilisateur d'une implémentation de s'affranchir de la question rébarbative de l'implémentation des
convertisseurs d'unité à unité.

En adoptant un périmètre limité, la présente spécification a également pour objectif de s'abstraire des spécificités des
langages de programmation de manière à ce que la représentation conceptuelle qu'elle propose puisse être aisément
implémentée dans plusieurs langages de programmation. Toutefois, la spécification est explicitement orientée objet.

La présente spécification n'inclut pas les contrôles d'homogénéité de conversion d'unités ni de cohérence des dimension
ni des grandeurs. La cohérence des conversions est laissée à l'utilisateur.

On se restreint à la représentation des conversions et des unités communes de manière à n'implémenter que des cas
relativement simples de conversions affines ou linéaires.

La conceptualisation des bases, des systèmes d'unités et des considérations relatives aux mesures n'entrent pas dans
le champ de cette spécification.

Sauf mention contraire relative aux éléments optionnels ou recommandés, le respect de la spécification abstraite est
impératif.

Dans les diagrammes de cette section, les méthodes en italique sont optionnelles. Leur implémentation est
\emph{recommandée} pour qui souhaite s'aligner sur la spécification d'implémentation qui les utilise.

\subsection{Convertisseurs}

Un convertisseur représente une conversion affine ou linéaire d'une unité à une autre. La méthode \emph{convert()} prend
en paramètre une valeur exprimée dans l'unité de départ et retourne son expression dans l'unité cible.

Les méthodes \emph{scale()} et \emph{offset()} retournent respectivement la pente et l'ordonnée à l'origine de la
formule de conversion linéaire ou affine et la méthode \emph{inverse()} retourne la conversion inverse de l'unité cible
vers l'unité de départ du convertisseur à partir duquel la méthode est appelée.

On \emph{recommande} que dans le convertisseur inverse du convertisseur inverse d'un convertisseur soit identique à ce
dernier au sens de l'instance.

La méthode \emph{linear()} est optionnelle. Elle construit un convertisseur qui ne garde que la partie linéaire du
convertisseur sur lequel la méthode est appelé. Lorsque cette méthode est implémentée, on \emph{recommande} que si le
convertisseur sur lequel elle est appelé est déjà linéaire, alors la méthode renvoie ce convertisseur lui-même au sens
de l'instance.

La méthode \emph{linearPow()} est optionnelle. Elle construit un convertisseur qui ne garde que la partie linéaire du
convertisseur sur lequel la méthode est appelé, mais en l'élevant à la puissance indiquée en paramètre. Lorsque cette
méthode est implémentée, on \emph{recommande} que si le convertisseur sur lequel elle est appelée est déjà linéaire
\emph{et} que la puissance indiquée en paramètre est égale à 1, alors la méthode renvoie ce convertisseur lui-même au
sens de l'instance.

La méthode \emph{concatenate()} produit un convertisseur dont la méthode \emph{convert()} est équivalente à l'évaluation
en premier lieu de la méthode \emph{convert()} du convertisseur indiqué en paramètre et en second lieu de la méthode
\emph{convert()} du convertisseur d'appel sur la valeur qui en résulte.

\begin{figure}[!h]
\begin{tikzpicture}
\umlclass{UnitConverter}{
  }{
  + scale() : decimal\\
  + offset() : decimal\\
  + inverse() : UnitConverter\\
  \umlvirt{+ linear() : UnitConverter}\\
  \umlvirt{+ linearPow(pow : decimal) : UnitConverter}\\
  + convert(value : decimal) : decimal\\
  + concatenate(converter : UnitConverter) : UnitConverter
}
\end{tikzpicture}
\caption{Contrat du convertisseur d'unités.}
\end{figure}

On \emph{recommande} que les implémentations fournissent des convertisseurs immuables et indiquent le cas échéant cette
caractéristique de manière explicite.

\subsection{Types d'unités}

Le type abstrait \emph{Unit} représente une unité de manière générale. Les unités concrètes sont de trois types: les
unités fondamentales (\emph{FundamentalUnit}) ne sont définies à partir d'aucune autre unité. Les unités transformées
(\emph{TransformedUnit}) sont définies par une unité de référence et un convertisseur qui représente la formule linéaire
ou affine qui permet de passer de l'une à l'autre. Les unités dérivées (\emph{DerivedUnit}) sont définies par une
collection d'unités chacune élevée à une puissance rationnelle, l'association d'une unité et d'une puissance rationnelle
constitutant un facteur (\emph{Factor}) de l'unité dérivée.

Les unités transformées et dérivées sont ainsi définies à partir d'autres unités et donc, directement ou indirectement,
de manière ultime à partir d'un jeu d'unités qui sont toutes fondamentales.

Toute unité est capable de fournir d'une part avec la méthode \emph{getConverterTo​()}, un convertisseur vers une unité
cible (sous réserve, dont l'appréciation est laissée à l'utilisateur, de la cohérence entre l'unité de départ et l'unité
cible du point de vue des unités fondamentales qui la composent) et d'autre part avec la méthode \emph{toBase()},
un convertisseur vers le jeu d'unités fondamentales qui la composent.

\begin{figure}[!h]
\begin{tikzpicture}
\umlclass[type=abstract, y=3]{Unit}{
  }{
  + getConverterTo​(unit: Unit) : UnitConverter\\
  + toBase() : UnitConverter
}
\umlclass[x=-3.5]{TransformedUnit}{
  }{
\umlvirt{+ reference(): Unit}\\
\umlvirt{+ toReference(): UnitConverter}
}
\umlclass[y=-3]{FundamentalUnit}{
  }{
}
\umlclass[x=3.5]{DerivedUnit}{
  }{
  \umlvirt{+ definition(): Factor[]}
}
\umlinherit{FundamentalUnit}{Unit}
\umlinherit{TransformedUnit}{Unit}
\umlinherit{DerivedUnit}{Unit}
\end{tikzpicture}
\caption{Contrats des différents types d'unités.}
\end{figure}

Les unités sont des instances immuables (impératif).

\subsection{Facteurs}

Un facteur représente l'association d'une unité à une exposant rationnel. La combinaison des facteurs permet de
construire les unités dérivées.

On \emph{suggère} que le concept d'unité spécialise celui de facteur de manière à considérer par polymorphysme qu'une
unité est un facteur d'elle-même élevée à la puissance 1. Cette représentation permet d'éviter la construction de
facteurs d'unités à la puissance 1 dans la définition de nombreuses unités dérivées courantes.


\begin{figure}[!h]
\begin{tikzpicture}
\umlclass[y=4]{Factor}{
  }{
    + dim() : Unit\\
    + numerator() : integer\\
    + denominator() : integer\\
    + power() : decimal
}
\umlclass{Unit}{
  }{
  + getConverterTo​() : UnitConverter\\
  + toBase() : UnitConverter\\
  + dim() : Unit \{ return this \}\\
  + numerator() : integer \{ return 1 \}\\
  + denominator() : integer \{ return 1 \}\\
  + power() : decimal \{ return 1. \}
}
\umlinherit{Unit}{Factor}
\end{tikzpicture}
\caption{Contrats du facteur et spécialisation suggérée par l'unité.}
\end{figure}

\subsection{Méthodes de construction}

On \emph{suggère} l'implémentation de méthodes de construction d'unités filles à partir d'une unité parente à la manière
des méthodes spécifiées par la JSR 363.

On suggère de manière non exhaustive, l'implémentation des méthodes suivantes:

\begin{itemize}
\item La méthode \emph{shift()} construit une unité transformée dont l'unité de référence est l'unité d'appel de la
méthode et le convertisseur à l'unité de référence est une translation dont la valeur est indiquée en paramètre.
\item La méthode \emph{scaleMultiply()} construit une unité transformée dont l'unité de référence est l'unité d'appel de
la méthode et le convertisseur à l'unité de référence est un agrandissement dont la valeur est indiquée en paramètre.
\item La méthode \emph{scaleDivide()} construit une unité transformée correspondant à un appel à \emph{scaleMultiply()}
avec un argument donc la valeur serait inversée.
\item La méthode \emph{factor()} construit un facteur dont l'unité est l'unité d'appel et la puissance est un rationnel
construit par le rapport du premier et du second paramètres.
\end{itemize}


\begin{figure}[!h]
\begin{tikzpicture}
\umlclass{Unit}{
  }{
  + getConverterTo​() : UnitConverter\\
  + toBase() : UnitConverter\\
  + shift(value : decimal) : TransformedUnit\\
  + scaleMultiply(value : number) : TransformedUnit\\
  + scaleDivide(value : number) : TransformedUnit\\
  + factor(numerator : integer, denominator : integer) : Factor
}
\end{tikzpicture}
\caption{Contrats du facteur et spécialisation suggérée par l'unité.}
\end{figure}

\section{Spécification d'implémentation}

La spécification d'implémentation est indicative. La spécification dans son ensemble peut être respectée par une
implémentation sans qu'elle s'aligne pour autant sur les choix d'implémentation exposés par cette section.

\subsection{Construction des convertisseurs}

Les convertisseurs sont construits par concaténation de la conversion de l'unité de départ vers les unités fondamentales
qui la composent et de la conversion inverse de l'unité cible vers ces mêmes unités fondamentales
(Algorithme~\ref{alg:getConverterTo​}).

\begin{algorithm}[!h]
\caption{Implémentation de Unit.getConverterTo​()}\label{alg:getConverterTo​}
\begin{algorithmic}
\Procedure{Unit.getConverterTo​}{target}\Comment{convertisseur de $this$ à $target$}
\State \textbf{return} target.toBase().inverse().concatenate(this.toBase())
\EndProcedure
\end{algorithmic}
\end{algorithm}

Pour cela, chaque type d'unité doit être capable de définir la conversion par laquelle exprimer une valeur dans les
unités fondamentales qui la composent en implémentant la méthode \emph{toBase()} de manière spécifique chacune à sa
nature.

La conversion d'une unité fondamentale vers le jeu d'unité fondamentale qui la constitue est l'identité. On
\emph{recommande} d'implémenter cette identité comme un convertisseur singleton.

\begin{algorithm}[!h]
\caption{Implémentation de FundamentalUnit.toBase()}\label{alg:fundamentalUnit:toBase}
\begin{algorithmic}
\Procedure{FundamentalUnit.toBase}{}
\State \textbf{return} identity()\Comment{un UnitConverter correspondant à l'identité.}
\EndProcedure
\end{algorithmic}
\end{algorithm}

La conversion d'une unité transformée vers le jeu d'untés fondamentales qui la composent consite à concaténer la
conversion vers son unité de référence à la conversion vers le jeu d'unités fondamentales qui composent cette dernière.

\begin{algorithm}[!h]
\caption{Implémentation de TransformedUnit.toBase()}\label{alg:transformedUnit:toBase}
\begin{algorithmic}
\Procedure{TransformedUnit.toBase}{}
\State \textbf{return} this.reference().toBase().concatenate(this.toReference())
\EndProcedure
\end{algorithmic}
\end{algorithm}

Enfin, la conversion d'une unité dérivée vers le jeu d'unités fondamentales qui la composent consiste à parcourir les
facteurs de définition de l'unité, d'en extraire la conversion vers le jeu d'unités fondamentales de l'unité du facteur,
de n'en garder que la part linéaire et de l'élever à la puissance du facteur. Chaque facteur fournissant ainsi un
convertisseur vers son jeu d'unités fondamentales, tous sont concaténés de manière à produire un convertisseur global.

L'algorithme ne garde que la partie linéaire des convertisseurs résultants des unités référencées par chaque facteur car
on considère que les translations doivent disparaître ainsi qu'on s'y attend, par exemple en convertissant des degrés
Celcius par mètre en degrés Kelvin par mètre.

\begin{algorithm}[!h]
\caption{Implémentation de DerivedUnit.toBase()}\label{alg:derivedUnit:toBase}
\begin{algorithmic}
\Procedure{DerivedUnit.toBase}{}

\State result : UnitConverter $\gets$ identity()
\For{factor $\in$ this.definition()}
\State toFactorBase $\gets$ factor.dim().toBase().linearPow(factor.power())
\State result $\gets$ toFactorBase.concatenate(result)
\EndFor

\State \textbf{return} result
\EndProcedure
\end{algorithmic}
\end{algorithm}

\section{Validation}

La conformité aux critères de validation est impérative au respect de la spécification. On ne donne cependant pas de
critère relatif à la précision des conversions au-delà des valeurs indiquées.

\end{document}
